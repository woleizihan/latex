\documentclass[psamsfonts]{amsart}

%-------Packages---------
\usepackage{amssymb,amsfonts}
\usepackage[all,arc]{xy}
\usepackage{enumerate}
\usepackage{mathrsfs}

%--------Theorem Environments--------
%theoremstyle{plain} --- default
\newtheorem{thm}{Theorem}[section]
\newtheorem{cor}[thm]{Corollary}
\newtheorem{prop}[thm]{Proposition}
\newtheorem{lem}[thm]{Lemma}
\newtheorem{conj}[thm]{Conjecture}
\newtheorem{quest}[thm]{Question}

\theoremstyle{definition}
\newtheorem{defn}[thm]{Definition}
\newtheorem{defns}[thm]{Definitions}
\newtheorem{con}[thm]{Construction}
\newtheorem{exmp}[thm]{Example}
\newtheorem{exmps}[thm]{Examples}
\newtheorem{notn}[thm]{Notation}
\newtheorem{notns}[thm]{Notations}
\newtheorem{addm}[thm]{Addendum}
\newtheorem{exer}[thm]{Exercise}

\theoremstyle{remark}
\newtheorem{rem}[thm]{Remark}
\newtheorem{rems}[thm]{Remarks}
\newtheorem{warn}[thm]{Warning}
\newtheorem{sch}[thm]{Scholium}

\makeatletter
\let\c@equation\c@thm
\makeatother
\numberwithin{equation}{section}

\bibliographystyle{plain}

%--------Meta Data: Fill in your info------
\title{Dirichlet Theorem}

\author{Ang Li}

\date{DEADLINE AUGUST 26, 2011}

\begin{document}

\begin{abstract}

This is a sample latex document with emphasis on using math mode and
equation environments.  You should use it as a template for your paper.
Remember that a first draft must be submitted to mentors at the very 
latest by August 15, with
no exceptions.  The completed paper must be submitted by August 26, unless 
permission for a later date has been obtained from the director of the program. 


\end{abstract}

\maketitle

\tableofcontents

\section{What does the table of contents command do?}

The table of contents command will automatically make a contents.
You must run tex at least twice for this to work.

\section{How do the environment commands work?}

``Environments'' are commands that are given using the \verb|\begin{}|
  and \verb|\end{}| syntax. In the preamble, you can see we've defined
  a bunch of theorem-type environments, including, for example ``defn'' To get a definition, 
you type:

\begin{defn}  This is how to define a definition.
\end{defn}

And for a theorem and its proof you would type:

\begin{thm}
This is the statement of a theorem.
\end{thm}
\begin{proof}
And this shows that the statement is correct.
\end{proof}

Note that the numbering is taken care of automatically, and that we've predefined a bunch of these sorts of environments to take care lemmas, corollaries and such in the header. 

Another useful kind of enviroment is the equation environment.  Equations
get numbered in sequence with statements, as for example

\begin{equation}  e = mc^2
\end{equation}

Note if you do not want a numbered equation, you can use the
environment ``equation*''
 like so:

\begin{equation*}
e=mc^2
\end{equation*}

There are plenty of other equation-type enviroments that allow you to
align several equations and such. The AMS's guide \cite{amsshort} is a
good place to start with these.

You can also typeset math directly in a paragraph by placing it within
dollar signs.  This is called ``math mode.''  For example: Let $e$
be energy, $m$ be momentum and $c$ be the speed of light.  Then
Einstein's famous equation says that $e=mc^2$.  This is useful, but
remember that it is harder to read inline math than displayed math. 

Remember that letters get put in a different font in math mode, so
whenever you are referencing a mathematical object you should always
put it in dollar signs.  For example, $f$ is a function, but f is just
a random letter.

Both \cite{notsoshort} and \cite{amsshort} have good lists of other symbols you can use in math mode.
These include greek letters ($\alpha, \beta, \Gamma, \Delta$),
operators ($\otimes, +, \sum$) and much more ($\leq, \diamond$).


\section{Xypic and diagrams}

If you want to draw diagrams, you should use xypic.  It's actually
much easier than it looks, and we've already included it in the header
above.  Here is an example. 
\[\xymatrix{
FX \ar[r]^-{Ff} \ar[d]_{\eta_X} & FY \ar[d]^{\eta_Y} \\
GX \ar[r]_-{Gf} & GY\\} \]

\subsection*{Acknowledgments}  It is a pleasure to thank my mentor, 
his/her name, for ....

\begin{thebibliography}{9}

\bibitem{amsshort}
Michael Downes.
Short Math Guide for \LaTeX.
ftp://ftp.ams.org/pub/tex/doc/amsmath/short-math-guide.pdf

\bibitem{May}
J. P. May.
A Concise Course in Algebraic Topology.
University of Chicago Press. 1999. 

\bibitem{notsoshort}
Tobias Oekiter, Hubert Partl, Irene Hyna and Elisabeth Schlegl.
The Not So Short Introduction to \LaTeXe.
http://tobi.oetiker.ch/lshort/lshort.pdf.

\end{thebibliography}

\end{document}

