\documentclass[psamsfonts]{amsart}

%-------Packages---------
\usepackage{amssymb,amsfonts}
\usepackage[all,arc]{xy}
\usepackage{enumerate}
\usepackage{mathrsfs}

%--------Theorem Environments--------
%theoremstyle{plain} --- default
\newtheorem{thm}{Theorem}[section]
\newtheorem{cor}[thm]{Corollary}
\newtheorem{prop}[thm]{Proposition}
\newtheorem{lem}[thm]{Lemma}
\newtheorem{conj}[thm]{Conjecture}
\newtheorem{quest}[thm]{Question}

\theoremstyle{definition}
\newtheorem{defn}[thm]{Definition}
\newtheorem{defns}[thm]{Definitions}
\newtheorem{con}[thm]{Construction}
\newtheorem{exmp}[thm]{Example}
\newtheorem{exmps}[thm]{Examples}
\newtheorem{notn}[thm]{Notation}
\newtheorem{notns}[thm]{Notations}
\newtheorem{addm}[thm]{Addendum}
\newtheorem{exer}[thm]{Exercise}

\theoremstyle{remark}
\newtheorem{rem}[thm]{Remark}
\newtheorem{rems}[thm]{Remarks}
\newtheorem{warn}[thm]{Warning}
\newtheorem{sch}[thm]{Scholium}

\makeatletter
\let\c@equation\c@thm
\makeatother
\numberwithin{equation}{section}

\bibliographystyle{plain}

\begin{document}
	\begin{abstract}
		Abstract here.
	\end{abstract}
	
	\title{Dirichlet's Theorem}
	\author{Ang Li}
	\date{}
	\maketitle
	
	\tableofcontents
	
	\section{Fourier analysis on $\mathbb{Z}(N)$ }
		\vspace{5mm}
		\subsection{The group $\mathbb{Z}(N)$}
			\begin{defn}
				A complex number $z$ is an $N^{\text{th}}$ \textbf{root of unity} if $z^N = 1$. We denote the set of all $N^{\text{th}}$ roots of unity by $\mathbb{Z}(N)$
			\end{defn}
		
			\begin{defn}
				Two integers $x$ and $y$ are \textbf{congruent modulo} $N$ if the difference $x-y$ is divisible by $N$, and we write $x \equiv y$ mod $N$.
			\end{defn}
		
			\begin{itemize}
				\item $x \equiv x$ mod $N$ for all integers $x$
				\item If $x \equiv y$ mod $N$, then $y \equiv x$ mod $N$
				\item If $x \equiv y$ mod $N$, and $y \equiv z$ mod $N$, then $x \equiv z$ mod $N$ 
			\end{itemize}	
			\vspace{2mm}
			Thus the relation $\equiv$ on $\mathbb{Z}$ is an equivalence relation. Let $R(x)$ denote the equivalence class, or residue class, of integer $x$. There are $N$ equivalence classes and each class has a unique representative between $0$ and $N-1$
		
			\begin{defn}
				The group of integers modulo $N$, sometimes denoted by $\mathbb{Z}/N\mathbb{Z}$, is $\{0,1,2.....N-1\}$.
			\end{defn}		
		\vspace{1mm}		
		
		\subsection{Fourier inversion theorem and Plancherel identity on $\mathbb{Z}(N)$}
			Let $e_n(x) = e^{2\pi inx}$\\
			$e_n(x+y) = e_n(x)+e_n(y)$\\
			On $\mathbb{Z}(N)$, the appropriate analogues are the $N$ functions $e_0$,...,$e_{N-1}$ defined by\\
			
			\begin{center}
				$e_l(k) = \zeta ^{lk} = e^{2\pi lk/N}$ for $l = 0,...,N-1$ and $k = 0,...N-1$,			
			\end{center}
			\vspace{2mm}
			where $\zeta = e^{2\pi ilk/N}$
			
			\begin{defn}
				The \textbf{Hermitian inner product} over a vector space is defined by
				\begin{center}
					$(F,G) = \sum_{k=0}^{N-1} F(k)\overline{G(k)}$
				\end{center}	
				and associated norm 
				\begin{center}
					$\|F\| = \sum_{k=0}^{N-1} |F(k)|^2$
				\end{center}
			\end{defn}
			
			\begin{lem}
				The family $\{e_0,...e_{N-1} \}$ is orthogonal. In fact,\\
				\begin{center}
					$(e_m,e_l)=\left\{\begin{array}{cl}
								N, & if \hspace{1mm} m=l,\\
								0, & if \hspace{1mm} m\neq l.
									\end{array}\right.
					$
				\end{center}
				
				\begin{proof}
					We have\\
					\begin{center}
						$
					(e_m,e_l)=\sum_{k=0}^{N-1} \zeta ^{mk} \zeta^{-lk}=\sum_{k=0}^{N-1} \zeta ^{(m-l)k}.			
						$
					\end{center}
					\vspace{2mm}
					If $m=l$, $\zeta ^{(m-l)k}=1$ for each $k$, and $(e_m,e_l)=N$. If $m\neq n$ then $q=\zeta ^{m-l}$ is not equal to 1, and\\
					
					\begin{center}
						$1+q+q^2+...+q^{N-1}=\frac{1-q^N}{1-q}=0$
					\end{center}
					\vspace{2mm}
					because $q^N=\zeta ^{(m-l)N=e^{2(m-l)\pi}}=1$
				\end{proof}
			\end{lem}
			
			\begin{defn}
				The $n^{\text{th}}$ \textbf{Fourier coefficient} of $F$ by\\
				\begin{center}
					$a_n=\sum_{k=0}^{N-1} F(k)e^{-2\pi ikn/N} $
				\end{center}
			\end{defn}
			
			\vspace{2mm}			
			
			\begin{thm}
				\emph{If $F$ is a function on $\mathbb{Z}(N)$, then\\
					\begin{center}
						$F(k)=\sum_{n=0}^{N-1} a_ne^{2\pi ink/N}$.
					\end{center}	
					Moreover,\\
					\begin{center}
						$\sum_{n=0}^{N-1}|a_n|^2 = \frac{1}{N} \sum_{k=0}^{N-1} |F(k)|^2$.
					\end{center}													
				}
				\vspace{2mm}				
				
				\begin{proof}
					We define $e_l^* = \frac{1}{\sqrt{N}} e_l$. Since the vector space $V$ of all complex-valued functions on $\mathbb{Z}(N)$ is $N$-dimensional, and from the lemma $\{ e_0,...e_{N-1}\}$ is orthogonal, $\{ e_0^*,...,e_{N-1}^*\}$ is an orthonormal basis for $V$. Hence for any $F \in V$ we have\\
					\begin{center}
						$F = \sum_{n=0}^{N-1} (F,e_n^*)e_n^*$ \hspace{2mm} and \hspace{2mm} $\|F\| = \sum_{n=0}^{N-1}|(F,e_n^*)|^2$
					\end{center}
					We also have\\
					\begin{center}
						$(F,e_n^*) = \sqrt{N}\sum_{k=0}^{N-1} F(k)e^{-2\pi ink/N} = \sqrt{N}a_n$
					\end{center}
					\vspace{2mm}
					Then\\
					\begin{center}
						$F(k) = \sum_{n=0}^{N-1} \sqrt{N} a_ne_n^*(k) = \sum_{n=0}^{N-1} a_ne^{2\pi ink/N} $
					\end{center}
					\vspace{2mm}
					Moreover,\\
					\begin{center}
						$\sum_{n=0}^{N-1} |a_n|^2 = \sum_{n=0}^{N-1} |(F,e_n^*)|^2 = \|F\|^2 = \frac{1}{N} \sum_{k=0}^{N-1} |F(k)|$
					\end{center}
				\end{proof}
			\end{thm}
			
			\vspace{5mm}
			
	\section{Fourier analysis on finite abelian groups}
		\vspace{5mm}
		\subsection{Abelian groups}
			\begin{defn}
				An \textbf{abelian group} (or commutative group) is a set $G$n together with a binary operation on pairs of elements of $G$, $(a,b) \longmapsto a \cdot b$, that satsfies the following conditions \\
				\begin{enumerate}
					\item \emph{Associativity} : $a \cdot (b \cdot c) = (a \cdot b)\cdot c$ \hspace{2mm} for all $a,b,c \in G$.
					\vspace{2mm}
					\item \emph{Identity} : There exists an element $u \in G$(often written as either 1 or 0) such that $a \cdot u = u \cdot a = a$ \hspace{2mm} for all $a \in G$.
					\vspace{2mm}
					\item \emph{Inverses} : For every $a \in G$, there exists an element $a^{-1} \in G$ such that $a \cdot a^{-1} = a^{-1} \cdot a = u$.
					\vspace{2mm}
					\item \emph{Commutativity} : For $a,b \in G$, we have $a \cdot b = b \cdot a$.
				\end{enumerate}
			\end{defn}
			
			\begin{defn}
				A \textbf{homomorphism} between two abelian groups $G$ and $H$ is a map $f:G \rightarrow H$ which satisfies the property\\
				\begin{center}
					$f(a \cdot b) = f(a) \cdot f(b)$,
				\end{center}
				\vspace{2mm}
				where the dot on the left-hand side is the operation in $G$, and the dot on the right-hand side the operation in $H$.
			\end{defn}
			
			\begin{defn}
				Two groups $G$ and $H$ are \textbf{isomorphic}, and write $G \approx H$, if there is a bijective homomorphism from $G$ to $H$.
			\end{defn}
			
			\begin{defn}
				In finite abelian group $G$, the \textbf{order} of $G$ is the number of elements in $G$, denoted by $|G|$.
			\end{defn}
			
			\begin{defn}
				If $G_1$ and $G_2$ are two finite abelian groups, their \textbf{direct product} $G_1 \times G_2$ is the group whose elements are pairs $(g_1,g_2)$ with $g_1 \in G_1$ and $g_2 \in G_2$. The operation in $G_1 \times G_2$ is them defined by\\
				\begin{center}
					$(g_1,g_2) \cdot (g_1^{\prime}, g_2^{\prime}) = (g_1 \cdot g_1^{\prime} , g_2 \cdot g_2^{\prime})$.
				\end{center}
			\end{defn}
			
			\vspace{2mm}
			
			Clearly, if $G_1$ and $G_2$ are two finite abelian groups, then so is $G_1 \times G_2$
			
			\begin{defn}
				An integer $n \in \mathbb{Z}(q)$ is a \textbf{unit} if there exists an integer $m \in \mathbb{Z}(q)$ so that\\
				\begin{center}
					$nm \equiv 1$ \hspace{3mm} mod $q$.
				\end{center}
				\vspace{2mm}		 						
				The set of all units in $\mathbb{Z}(q)$ is denoted by $\mathbb{Z}^*(q)$.
			\end{defn}
			
		\subsection{Characters}
			\begin{defn}
				Let $G$ be a finite abelian group and $S^1$ the unit circle in the complex plane. A \textbf{character} on $G$ is a complex-valued function $e : G \rightarrow S^1$ which satisfies the following condition:\\
				\begin{center}
					$e(a \cdot b) = e(a) \cdot e(b)$ \hspace{3mm} fro all $a,b \in G$
				\end{center}
				The \textbf{trivial} or \textbf{unit character} is defined by $e(a) = 1$ for all $a \in G$
			\end{defn}
			
			If $G$ is a finite abelian group, we denote by $\hat{G}$ the set of all characters of $G$.
			
			\begin{lem}
				The set $\hat{G}$ is an abelian group under multiplication defined by\\
				\begin{center}
					$(e_1 \cdot e_2)(a) = e_1(a) \cdot e_2(a)$ \hspace{2mm} for all $a \in G$.
				\end{center}
			\end{lem}			
			
			\vspace{2mm}
			
			\begin{lem}
				Let G be a finite abelian group, and $e : G \rightarrow \mathbb{C} - \{0\}$ a multiplicative function, namely $e(a \cdot b) = e(a)e(b)$ for all $a,b \in G$. Then e is a character.
				\begin{proof}
					The group $G$ is finite, then $|e(a)|$ is bounded above and below as as $a$ ranges over $G$. Since $|e(b^n)| = |e(b)|^n$, $|e(b)| = 1$ for all $b \in G$
				\end{proof}
			\end{lem}
			
		\subsection{The orthogonality relations}
			\begin{lem}
				If $e$ is a non-trivial character of the group $G$, then $\sum_{a \in G} e(a) = 0$.
				\begin{proof}
					Choose $b \in G$ such that $e(b) \neq 1$. Then\\
						\begin{center}
							$e(b)\sum_{a \in G} e(a) = \sum_{a \in G} e(b)e(a) = \sum_{a \in G} e(ab) = \sum_{a \in G} e(a)$.
						\end{center}
						\vspace{2mm}
					Therefore $\sum_{a \in G} e(a) = 0$.
				\end{proof}
			\end{lem}		
		
			\begin{thm}
				The characters of $G$ form an orthonormal family with respect to the Hermitian inner product.
				\begin{proof}
					Since $|e(a)| = 1$ for any character, we have\\
					\begin{center}
						$(e,e) = \frac{1}{|G|} \sum_{a \in G} e(a)\overline{e(a)} = \frac{1}{|G|} \sum_{a \in G} |e(a)|^2 = 1$.
					\end{center}
					\vspace{2mm}
					If $e \neq e^{\prime}$ and both $e$ and $e^{\prime}$ are characters, we must prove that $(e,e^{\prime}) = 0$.
					$e \neq e^{\prime}$ implies that $e(e^{\prime})^{-1}$ is non-trivial. The lemma shows that\\
					\begin{center}
						$\sum_{a \in G} e(a)(e^{\prime}(a))^{-1} = 0$.
					\end{center}
					\vspace{2mm}
					
					Since $(e^{\prime}(a))^{-1} = \overline{e^{\prime}(a)}$, the theorem is proved.
				\end{proof}
			\end{thm}			
		\vspace{1mm}
		
		\subsection{Characters as a total family}
			\begin{defn}
				A linear transformation $T : V \rightarrow V$ is \textbf{unitary} if it preserves the inner product, $(Tv,Tw) = (v,w)$ for all $v,w \in V$
			\end{defn}		
		
			\begin{thm}
				(spectral theorem)\\
				Any unitary transformation on a finite-dimensional space is diagonalizable. In other words, there exists a basis $\{v_1,...,v_d\}$(eigenvectors) of $V$ such that $T(v_i)=\lambda_iv_i$, where $\lambda_i \in \mathbb{C}$ is the eigenvalue attached to $v_i$.
			\end{thm}
			
			\begin{lem}
				Suppose $\{T_1,...,T_k\}$ is a commuting family of unitary transformations on the finite-dimensional inner product space V; that is,\\
				\begin{center}
					$T_iT_j = T_jT_i$ \hspace{2mm} for all $i,j$.
				\end{center}
				\hspace{2mm}
				Then $T_1,...,T_k$ are simultaneously diagonalizable. In other words, there exists a basis for $V$ which consists of eigenvectors for every $T_i$.
			\end{lem}
						
			\begin{thm}
				The characters of a finite abelian group $G$ form a basis for the vector space of functions on $G$.
			\end{thm}
			
		\subsection{Fourier inversion and Plancherel formula}
			\begin{defn}
				Given a finite abelian group $G$ and a function $f$ on $G$, define the \textbf{Fourier coefficient} of $f$ with respect to character $e$ of $G$, by\\
				\begin{center}
					$\hat{f}(e) = (f,e) = \frac{1}{|G|} \sum_{a \in G} f(a)\overline{e(a)}$,
				\end{center}
				\vspace{2mm}
				and the \textbf{Fourier series} of f as\\
				\begin{center}
					$f \sim \sum_{e \in \hat{G}} \hat{f}(e)e$
				\end{center}
			\end{defn}
			\vspace{2mm}
			\begin{thm}
				Let $G$ be a finite abelian group. The characters of $G$ form an orthonormal basis for the vector space $V$ of functions on $G$ equipped with the Hermitian inner product.In particular, any function $f$ on $G$ is equal to its Fourier series\\
				\begin{center}
					$f = \sum_{e \in G} \hat{f}(e)e$.
				\end{center}
				
				\begin{proof}
					Since the characters of the finite abelian group $G$ forms an orhtonormal basis for the vector space $V$ of functions on $G$, then\\
					\begin{center}
						$f = \sum_{e \in \hat{G}} c_ee$
					\end{center}
					\vspace{2mm}
					for some set of constants $c_e$. Also, by because the orthogonality, we have\\
					\begin{center}
						$(f,e) = c_e = \hat{f}(e)$.
					\end{center}
					\vspace{2mm}
					Therefore, $f = \sum_{e \in \hat{G}} \hat{f}(e) e$.
				\end{proof}
			\end{thm}
			
			\begin{thm}
				(the Parseval-Plancherel formula)
				If $f$ is a function on $G$, then $\|f\|^2 = \sum_{e \in \hat{G}} |\hat{f}()e|^2$.
				\begin{proof}
					Since the characters of $G$ form an orthonormal basis for vector space $V$, and $(f,e) = \hat{f}(e)$, we have\\
					\begin{center}
						$\|f\|^2 = (f,f) = \sum_{e \in \hat{G}} (f,e)\overline{\hat{f}()e} = \sum_{e \in \hat{G}} |\hat{f}(e)|^2$
					\end{center}										
				\end{proof}
			\end{thm}
	
	\section{Elementary number theory}
		\vspace{5mm}
		\subsection{The fundamental theorem of arithmetic}
			\begin{thm}
				(Euclid's algorithm) For any integers $a$ and $b$ with $b > 0$, there exists unique integers $q$ and $r$ with $0 \leq r < b$ such that\\
				\begin{center}
					$a = qb +r$.
				\end{center}
			\end{thm}
							
			\begin{defn}
				An integer $a$ \textbf{divides} $b$ if there exists another integer $c$ such that $ac = b$; we then write $a|b$ and say that $a$ is a \textbf{divisor} of $b$. A \textbf{prime number} is a positive integer greater than $1$ that has no positive divisors besides $1$ and itself. 
			\end{defn}
			
			\begin{defn}
				The \textbf{greatest common divisor} of two positive integers $a$ and $b$ is the largest integer that divides both $a$ and $b$. Two positive integers are \textbf{relatively prime} if their greatest common divisor is $1$.
			\end{defn}			
			
			\begin{thm}
				If $gcd(a,b) = d$, then there exist integers $x$ and $y$ such that\\
				\begin{center}
					$ax + by = d$
				\end{center}								
				\vspace{2mm}
				
				\begin{proof}
					Consider the set $S$ of all positive integers of the form $ax+by$ where $x,y \in \mathbb{Z}$, and let $s$ be the smallest element in $S$. Claim that $s = d$. There exists integers $x$ and $y$ such that\\
					\begin{center}
						$ax+by = s$
					\end{center}
					\vspace{2mm}
					
					Clearly, any divisor of $a$ and $b$ divides $s$, so we have $d \leq s$. BY Euclid's algorithm, we can write $a = qr + r$ with $0 \leq r < s$. By $ax+by = s$, we have $qax + qby = qs = a - r$. Hence, $r = a(1-qx) + b(-qy)$. Since $s$ is the minimal in $S$, we have $r = 0$. Therefore, $s|a$ and similarly $s|b$. Then $s = d$.
				\end{proof}
			\end{thm}
\end{document}