\documentclass[12pt]{article}

\usepackage{amssymb,amsmath,amsthm} 
\usepackage[margin=1.25in]{geometry}
\usepackage{graphicx,ctable,booktabs}
\usepackage{enumerate}

\makeatletter
\newenvironment{problem}{\@startsection
       {section}
       {1}
       {-.2em}
       {-3.5ex plus -1ex minus -.2ex}
       {2.3ex plus .2ex}
       {\pagebreak[3]%forces pagebreak when space is small; use \eject for better results
       \large\bf\noindent{Problem }
       }
       }
       {%\vspace{1ex}\begin{center} \rule{0.3\linewidth}{.3pt}\end{center}}
       \begin{center}\large\bf \ldots\ldots\ldots\end{center}}
\makeatother

\usepackage{fancyhdr}
\pagestyle{fancy}
\lhead{Redmond McNamara}
\chead{} 
\rhead{\thepage} 
\lfoot{\small\scshape Algorithms in Finite Groups} 
\cfoot{} 
\rfoot{\footnotesize Problem Set 3} 
\renewcommand{\headrulewidth}{.3pt} 
\renewcommand{\footrulewidth}{.3pt}
\setlength\voffset{-0.25in}
\setlength\textheight{648pt}

\newcommand{\fff}{\mathbb F}
\newcommand{\zzz}{\mathbb Z}

\begin{document}

	\title{Algorithms in Finite Groups}
	\author{MATH 37500: L\'aszl\'o Babai \\
       	 Scribe: Ang Li} 
	\date{10/07/14}

	\maketitle

	\theoremstyle{definition}
	\newtheorem{ex}{Exercise}[section]
	\newtheorem{definition}[ex]{Definition}
	\setcounter{section}{1}

	\begin{ex}
		$S_k^{(2)} \leq S_{k \choose{2}}$ is primitive for $k \geq 5$.
	\end{ex}

	\begin{ex}
		Peterson's graph is isomorphic to the line graph of $K_5$.
	\end{ex}
	
	\begin{ex}
		Automorphism group of Dodecahedron is not isomorphic to $S_5$.
	\end{ex}
	
	\begin{ex}
		$\mathrm{Aut}^+(Dodecahedron) \cong A_5$ and $\mathrm{Aut}(Dodecaheron) \cong A_5 \times C_2$.
	\end{ex}
	
	\begin{ex}
		If $A$ is an orientation preserving congruence of $\mathbb{R}^3$, then $A \in \mathrm{SO}(3)$, which is the set of orthogonal matrices of determinant $1$. Hence $A$ is a rotation.
	\end{ex}
	
	\begin{ex}
		Given $A \in \mathrm{O}(3) \setminus \mathrm{SO}(3)$. Then $A$ can be represented by rotation and reflection in a plane perpendicular to the rotation axis.
	\end{ex}
	
	\begin{ex}
		Realize $4$-cycles in $\mathrm{Aut}(Tetrahedron)$ as rotational reflection.
	\end{ex}
	
	\begin{ex}
		$\mathrm{Aut}(Q_n)$ includes $\mathbb{Z}_2^n$ and $\mathbb{Z}_2^n \lhd \mathrm{Aut}(Q_n)$.
	\end{ex}
	
	\begin{ex}
		$\mathrm{Aut}(Q_n) / \mathbb{Z}_2^n \cong S_n$.
	\end{ex}
	
	\begin{ex}
		$\mathrm{Aut}(Q_n) \cong \mathbb{Z}_2^n \wr S_n$.
	\end{ex}
	
	\begin{ex}
		Is $\mathrm{Aut}(Q_n)$ primitive?
	\end{ex}
	
	\begin{ex}
		If $A$ is primitive and not of prime order and $B$ is primitive. Then $A \wr B$ in product action is primitive.
	\end{ex}
	
	\begin{ex}
		$G_{x \to y} = \{g: x^g = y \} = G_x \cdot g_0 = g_0 \cdot G_y$, where $g_0 \in G$ such that $x^{g_0} = y$.
	\end{ex}
	
	\begin{ex}
		$[R_G, L_G] = 1$. $R_G, L_G$ centralize each other.
	\end{ex}
	
	\begin{ex}
		$C_{\mathrm{Sym}(G)}(R_G) = L_G$.
	\end{ex}
	
	\begin{ex}
		$L_G$ and $R_G$ are permutationally isomorphic. In particular, every regular permutation group any regular permutation group $H \leq \mathrm{Sym}(G)$ is isomorphic to $R_G$.
	\end{ex}
	
	\begin{ex}
		If $G$ is semiregular, then $C_{\mathrm{Sym}(\Omega)}(G)$ is transitive.
	\end{ex}
	
	\begin{ex}
		If $G$ is transitive, then $C_{\mathrm{Sym}(\Omega)}(G)$ is semiregular.
	\end{ex}
	
	\begin{ex}
		$G$ transitive and $N \lhd G$, then orbits of $N$ form a $G$-invariant partition of $\Omega$.
	\end{ex}
	
	\begin{ex}
		$G$ primitive and has a normal subgroup $N$ which is not trivial, then $N$ is transitive.
	\end{ex}
	
	\begin{ex}
		Transitive abelian group is regular.
	\end{ex}
	
	\begin{ex}
		Give $H \leq G$, there exists a unique permutation action of $G$ such that $G_x$ corresponds to $H$.
	\end{ex}
	
	\begin{ex}
		The kernel of this permutation action is $\mathrm{Core}(H) = \cap_{g \in G}H^g$.
	\end{ex}

\end{document}