\documentclass[psamsfonts]{amsart}

%-------Packages---------
\usepackage{amssymb,amsfonts}
\usepackage[all,arc]{xy}
\usepackage{enumerate}
\usepackage{mathrsfs}

%--------Theorem Environments--------
%theoremstyle{plain} --- default
\newtheorem{theorem}{Theorem}[section]
\newtheorem{cor}[theorem]{Corollary}
\newtheorem{prop}[theorem]{Proposition}
\newtheorem{lem}[theorem]{Lemma}
\newtheorem{conj}[theorem]{Conjecture}
\newtheorem{quest}[theorem]{Question}
\newtheorem{obs}[theorem]{Observation}

\theoremstyle{definition}
\newtheorem{defn}[theorem]{Definition}
\newtheorem{defns}[theorem]{Definitions}
\newtheorem{con}[theorem]{Construction}
\newtheorem{exmp}[theorem]{Example}
\newtheorem{exmps}[theorem]{Examples}
\newtheorem{notn}[theorem]{Notation}
\newtheorem{notns}[theorem]{Notations}
\newtheorem{addm}[theorem]{Addendum}
\newtheorem{exer}[theorem]{Exercise}

\theoremstyle{remark}
\newtheorem{rem}[theorem]{Remark}
\newtheorem{rems}[theorem]{Remarks}
\newtheorem{warn}[theorem]{Warning}
\newtheorem{sch}[theorem]{Scholium}

\makeatletter
\let\c@equation\c@theorem
\makeatother
\numberwithin{equation}{section}

\bibliographystyle{plain}

\begin{document}
	\section{Introduction}
	Sensitivity is a measure on Boolean function such that something something. A major open question in this field is the sensitivity conjecture, which says block sensitivity is polynomially bounded by sensitivity. 
	\begin{•}	
	The largest known gap is quadratic and this is due to Rubinstein. Turan has shown that for a graph property on $v$ vertices, $s(f) = \Omega(v)$, which implies the largest gap possible for graph properties is quadratic. In this paper, we prove that the bounds are tight. Specifically, we demonstrate a graph property $f$ with $\Theta(v)$ sensitivity and $\Theta(v^2)$ block sensitivity. Moreover, we generalize this property to $k$-uniform hypergraphs and produce quadratic gaps for all $k$ even.

	For $k$ odd, we don't have this nice quadratic gap. Instead, we make indirect progress by demonstrating $k$-uniform hypergraph properties with $O(v^{k/2})$ sensitivity. Conjecture by Kenyon and Kutin says that for a "nice" boolean function $f$, $s^0(f)s^1(f) = \Omega(n)$. However, Chakraborty showed that this was not true by giving a counterexample with sensitivity $\Theta(n^{1/3})$. Here, we conjecture that the inequality is holds for $k$-uniform hypergraph properties. Assuming this conjecture, we have a lower bound of $\Omega(v^{k/2})$ for the sensitivity of $k$-uniform hypergraphs and we give graph properties for $k$ odd that show that this conjectured lower bound is tight.
	Let $f: \{0,1\}^n \to {0,1}$ be a Boolean function. For any input $x \in \{0,1 \}^n$, let $x^{(i)}$ be $x$ with $i^{th}$ bit flipped. Then the $i ^ {th}$ bit is said to be \textit{sensitive} for $f$ if $f(x) \neq f(x^{(i)})$. The sensitivity of $f$ on an input $x$, denoted by $s(f,x)$ is the number of sensitive bits for $f$ on $x$.
		\begin{defn}
			The \textit{sensitivity} of a Boolean function $f$, denoted by $s(f)$ is the maximum of $s(f,x)$ over all $x \in \{0,1\}^n$.
		\end{defn}
		Similarly, given $x \in \{0,1\}^n$ and $B \subset [n]$, let $x^{B}$ be $x$ with $i^{th}$ bit flipped for any $i \in B$. Then the "block" $B$ is sensitive for $f$ on $x$ if $f(x) \neq f(x^B)$. And the \textit{block sensitivity} of $f$ on $x$, denoted by $bs(f,x)$ is the maximum number of \textit{pairwise disjoint} sensitive blocks of $f$ on $x$.
		\begin{defn}
			The \textit{block sensitivity} of a Boolean function $f$, denoted by $bs(f)$ is the maximum of $bs(f,x)$ over all $x \in \{0,1\}^n$.
		\end{defn}
		Obviously, for any Boolean function $f$, we have 
		\begin{equation}
			bs(f) \geq s(f). \nonumber
		\end{equation}
		
		\begin{defn}
			A Boolean function $f: \{0,1\}^{v \choose{2}} \to \{0,1\}$ is called a \textit{graph property} if for every input $x = (x_{(1,2)},...,x_{(n-1,n)})$ and every permutation $\sigma \in S_v$, we have
			\begin{equation}
				f(x_{(1,2)},...,x_{(n-1,n)}) = f(x_{(\sigma (1), \sigma (2))},...,x_{(\sigma (n-1), \sigma (n))}). \nonumber
			\end{equation}
		\end{defn}
		Similarly, we can define $k$-uniform hypergraph property.
			\begin{defn}
				A Boolean function $f: \{0,1\}^{v \choose{k}} \to \{0,1\}$ is called a \textit{$k$-uniform hypergraph property} if for every input $x = (x_{(1,2,...,k)},...,x_{(n-k+1,...,n-1,n)})$ and every permutation $\sigma \in S_v$, we have
			\begin{equation}
				f(x_{(1,2,...,k)},...,x_{(n-k+1...,n-1,n)}) = f(x_{(\sigma (1), \sigma (2),..., \sigma(k))},...,x_{(\sigma (n-k+1),...,\sigma (n-1), \sigma (n))}). \nonumber
			\end{equation}
			\end{defn}
\end{document}