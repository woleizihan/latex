\documentclass[psamsfonts]{amsart}

%-------Packages---------
\usepackage{amssymb,amsfonts}
\usepackage[all,arc]{xy}
\usepackage{enumerate}
\usepackage{mathrsfs}

%--------Theorem Environments--------
%theoremstyle{plain} --- default
\newtheorem{theorem}{Theorem}[section]
\newtheorem{claim}[theorem]{Claim}
\newtheorem{cor}[theorem]{Corollary}
\newtheorem{prop}[theorem]{Proposition}
\newtheorem{lem}[theorem]{Lemma}
\newtheorem{conj}{Conjecture}
\newtheorem*{conj*}{Conjecture}
\newtheorem{quest}[theorem]{Question}
\newtheorem{obs}[theorem]{Observation}
\newtheorem{oprob}[theorem]{Open Problem}

\theoremstyle{definition}
\newtheorem{defn}[theorem]{Definition}
\newtheorem{defns}[theorem]{Definitions}
\newtheorem{con}[theorem]{Construction}
\newtheorem{exmp}[theorem]{Example}
\newtheorem{exmps}[theorem]{Examples}
\newtheorem{notn}[theorem]{Notation}
\newtheorem{notns}[theorem]{Notations}
\newtheorem{addm}[theorem]{Addendum}
\newtheorem{exer}[theorem]{Exercise}
\newtheorem{prob}[theorem]{Problem}

\theoremstyle{remark}
\newtheorem{rem}[theorem]{Remark}
\newtheorem{rems}[theorem]{Remarks}
\newtheorem{warn}[theorem]{Warning}
\newtheorem{sch}[theorem]{Scholium}

\makeatletter
\let\c@equation\c@theorem
\makeatother
\numberwithin{equation}{section}

\bibliographystyle{plain}
%\;\;\makebox[0pt]{$\top$}\makebox[0pt]{$\cap$}\;\

\begin{document}

	\title{On Sensitivity of $k$-uniform Hypergraph Properties}

	\author{Joshua Biderman, Kevin Cuddy, Ang Li, Min Jae Song}
	\date{\today}
	\begin{abstract}
		 In this paper we present a graph property with sensitivity $\Theta(\sqrt{n})$, where $n={v\choose2}$ is the number of variables, and generalize it to a $k$-uniform hypergraph property with sensitivity $\Theta(\sqrt{n})$, where $n={v\choose k}$ is again the number of variables. This yields the smallest sensitivity yet achieved for a $k$-uniform hypergraph property. We then show that, for even $k$, there is a $k$-uniform hypergraph property that demonstrates a quadratic gap between sensitivity and block sensitivity. This matches the previously known largest gap found by Rubinstein (1995) for a general Boolean function and Chakraborty (2005) for a cyclically invariant Boolean function, and is the first known example of such a gap for a graph or hypergraph property.
	\end{abstract}
	\maketitle
	
	\tableofcontents
	
	\section{Introduction}
The sensitivity of a Boolean function $f: \{0,1\}^{n} \rightarrow \{0,1\}$ is the maximum over all inputs $x$ of the number of $i$ such that flipping the $i$-th bit of $x$ flips the value of $f(x)$. Block sensitivity, introduced by Nisan \cite{N}, is the maximum over all inputs $x$ of the maximum number of disjoint blocks of variables such that flipping all the bits in any single block flips the value of $f(x)$. We present these definitions with precise mathematical notation in section 2.  It is clear from the definition that for all $f$, $s(f) \leq bs(f)$, since we can just consider a partition of the input into blocks of size one.  Whether there is a polynomial bound in the other direction is a long-standing open question and is known as the Sensitivity Conjecture. In other words, do there exist $a,b\in\mathbb{R}$ such that $\forall f:\{0,1\}^n\to\{0,1\},  bs(f)<a\cdot s(f)^b$? A stronger conjecture by Nisan and Szegedy \cite{NS} is that $bs(f) \leq O(s(f)^{2})$. Currently, the best known upper bound on block sensitivity is exponential in sensitivity, as given by Kenyon and Kutin \cite{KK}; their bound is $O(\frac{e}{\sqrt{2\pi}}e^{s(f)}\sqrt{s(f)})$.

The Sensitivity Conjecture motivates the search for Boolean functions with large gaps between sensitivity and block sensitivity. The largest known gap is quadratic and is due to Rubinstein \cite{R}. Chakraborty \cite{C} explained how a minor modification to Rubinstein's function yields a cyclically invariant Boolean function also demonstrating a quadratic gap. This result raises the question: can even more symmetric functions, specifically graph properties and uniform hypergraph properties, demonstrate quadratic (or greater) gaps between sensitivity and block sensitivity? In this paper, we explore the sensitivity of $k$-uniform hypergraph properties and consider the following three conjectures of Laszlo Babai (all three conjectures are for every $k \in \mathbb{N}, k\geq 2$):

\begin{conj}
There exists a $k$-uniform hypergraph property $f$ such that $$s(f)=O(\sqrt{n})=O(v^{k/2}).$$
\end{conj}

\begin{conj}
There exists a $k$-uniform hypergraph property $f$ such that $s(f)$ and $bs(f)$ display a quadratic gap. That is,  $bs(f)=\Omega(s(f)^2)$.
\end{conj}

\begin{conj}
For every $k$-uniform hypergraph property $f,$ $$s(f)=\Omega(\sqrt{n})=\Omega(v^{k/2}).$$
\end{conj}

In Section 3, we answer Conjecture 1 in the affirmative by demonstrating, for every $k \in \mathbb{N}$, a $k$-uniform hypergraph property with sensitivity $s(f)=\Theta(\sqrt{n})=\Theta(v^{k/2})$. In Section 4, we partially answer Conjecture 2 in the affirmative by demonstrating a $k$-uniform hypergraph property with a quadratic gap between senstivity and block sensitivity for every even $k$.  However, for odd $k,$ the best we have accomplished is a gap of $\frac{2k}{k+1}$ in the exponent.  That is, for all odd $k$, we give a graph property $f$ such that 
$$bs(f)=\Theta(s(f)^{\frac{2k}{k+1}}).$$

Interestingly, in the case of $k=2$, we know that the function given in Section 4 is optimal in the sense of giving the largest gap between sensitivity and block sensitivity.  Turan \cite{T} has shown that for a graph property $f$ on $v$ vertices, $s(f) = \Omega(\sqrt{n})=\Omega(v)$. This also answers Conjecture 3 in the affirmative for the case of $k=2$, but for other values of $k$, the conjecture remains open. 

	\section{Preliminaries}
		We first introduce some complexity measures that can be used to analyze Boolean functions.

	\subsection{Definitions}
\begin{defn}\label{sensitivity}
For a Boolean function $f: \{0,1\}^n \to \{0,1\}$ and an input string $x \in \{0,1\}^{n}$, the \textbf{sensitivity of $\boldsymbol{f}$ at $\boldsymbol{x}$}, denoted as $\boldsymbol{s(f,x)},$ is $$s(f,x) =|\{i \in [n] \: | \: f(x) \neq f(x^{i})\}|$$ where 
$\lbrack n \rbrack$ is the set 
$\{1,2,\ldots ,n\}$ and 
$x^{i}$ is the string $x$ with the $i^{th}$ bit flipped, 
that is, changed either from 0 to 1 or from 1 to 0.  So the sensitivity of $f$ at $x$ is the number of single-bit changes to $x$ that change $f(x)$.  
\\
The \textbf{sensitivity of $\boldsymbol{f}$} or $\boldsymbol{s(f)}$ is then $$s(f) = \max\limits_{x \in \{0,1\}^{n}} s(f,x).$$
\end{defn}

There is a similar measure called block sensitivity, introduced by Nisan \cite{N}.  
\begin{defn}\label{blocksensitivity}
A \textbf{sensitive block for $\boldsymbol{f}$ at $\boldsymbol{x}$} is a subset $B \subseteq [n]$ such that 
$f(x) \neq f(x^{B})$, where $x^{B}$ is the string $x$ with all bits whose indices are in $B$ flipped.
For example, if $n=3, \: B_{1} = \{1,2\},$ and $x=101$, then $x^{B_{1}}$ is $011$.  
The \textbf{block sensitivity of $\boldsymbol{f}$ at $\boldsymbol{x}$} is then
$$bs(f,x) = \max_{B_{1}, \ldots , B_{k}} k$$ 
where the maximum is taken over all sets of disjoint sensitive blocks for $f$ at $x$.  This is the maximum number of disjoint sensitive blocks for $f$ at $x$.
The \textbf{block sensitivity of $\boldsymbol{f}$} or $\boldsymbol{bs(f)}$ is then defined to be
$$bs(f) = \max\limits_{x \in \{0,1\}^{n}} bs(f,x).$$
\end{defn}

Clearly, for any Boolean function $f$, we have 
		\begin{equation}
			s(f) \leq bs(f). \nonumber
		\end{equation}
		

\begin{defn}\label{oneandzero}
It will be useful when presenting proofs to use the notation $\boldsymbol{s^{0}(f)}$ for the sensitivity when changing $f$'s value from 0 to 1 
and $\boldsymbol{s^{1}(f)}$ for the sensitivity when changing 
the value from 1 to 0.  Sensitivity is then $\max \{ s^{0}(f), s^{1}(f)\}$.  The same notation will be used for block sensitivity when required.  
\end{defn}

\begin{defn}
Two complexity measures on Boolean functions $A(f)$, $B(f)$ are called \textbf{polynomially related} if there exist $n$, $m$, $s$, $t \in \mathbb{N}$ such that 
for any Boolean function $f$, $s \cdot A(f)^{n} \leq B(f)$ and $t \cdot B(f)^{m} \leq A(f)$.  
\end{defn}

As stated before, $s(f) \leq bs(f)$, but a polynomial bound in the other direction is unknown.  

		\begin{defn}
	A Boolean function $f: \{0,1\}^{v \choose{2}} \to \{0,1\}$ is called a \textbf{graph property} if for every input $x = (x_{(1,2)},...,x_{(n-1,n)})$ and every permutation $\sigma \in S_v$, we have
			\begin{equation}
				f(x_{(1,2)},x_{(1,3)},...,x_{(n-1,n)}) = f(x_{(\sigma (1), \sigma (2))},...,x_{(\sigma (n-1), \sigma (n))}). \nonumber
			\end{equation}
		\end{defn}
Here the Boolean string $x$ is interpreted as a graph in the following manner: $x_{(1,2)}$ is 1 if there is an edge connecting vertex 1 and vertex 2, and it is 0 if there is no such edge.  
The labelling of the vertices is of course arbitrary; thus we could have two distinct Boolean strings representing the ``same'' graph (two isomorphic graphs).  The length of the input string 
must be $\{v \choose{2}\}$ for some $v \in \mathbb{N}$ because this is how many edge relations must be specified to define a graph on $n$ vertices.  A graph property $f$ thus satisfies $f(G) = f(H)$ whenever $G$ and $H$ are isomorphic graphs.	Similarly, we can define $k$-uniform hypergraph property.
			\begin{defn}
				A Boolean function $f: \{0,1\}^{v \choose{k}} \to \{0,1\}$ is called a \textbf{$k$-uniform hypergraph property} if for every input $x = (x_{(1,2,...,k)},...,x_{(n-k+1,...,n-1,n)})$ and every permutation $\sigma \in S_v$, we have
			\begin{equation}
				f(x_{(1,2,...,k)},...,x_{(n-k+1...,n-1,n)}) = f(x_{(\sigma (1), \sigma (2),..., \sigma(k))},...,x_{(\sigma (n-k+1),...,\sigma (n-1), \sigma (n))}). \nonumber
			\end{equation}
			\end{defn}
Again, this means that $f$ takes on the same value for isomorphic hypergraphs.

	\subsection{Previous Results}

		The largest known gap between sensitivity and block sensitivity is quadratic, demonstrated by Rubinstein's function [4].
	\begin{exmp} ({\it Rubinstein's function})
		Let $n = k^2$ for some even $k$. We partition the $n$ variables into $k$ consecutive blocks of $k$ variables. We denote the blocks by $B_1,...,B_k$. Let $f(x) = 1$ if and only if there exists a block $B_i$ such that exactly two consecutive bits $x_{(i-1)k+j}, x_{(i-1)k+(j+1)} \in B_i$ take on the value 1 and all the other bits in $B_i$ are 0. The sensitivity of $f$ is $2k$ and the block sensitivity is $k^2/2$.
	\end{exmp}

	Chakraborty modified Rubinstein's function and constructed a cyclically invariant Boolean function with a quadratic gap between sensitivity and block sensitivity.
\begin{defn} Let $x = x_1x_2...x_n$ where each $x_i \in \{0,1\}$. Then for $0 < l < n$, the binary string $x_{l+1}x_{l+2}...x_nx_1...x_l$ is a \textbf{cyclic shift} of $x$.
\end{defn}

	\begin{exmp}
		Let $g : \{0,1\}^{k^2} \rightarrow \{0,1\}$ be Rubinstein's function. Define $f : \{0,1\}^{k^2} \rightarrow \{0,1\}$ as $f(x) = 1$ if and only if $g(x') = 1$ for some $x'$ which is a cyclic shift of $x$. $f$ has sensitivity $2k$ and block sensitivity $k^2/2$.
	\end{exmp}
	Cook et al. \cite{CDR} proved that for a CREW PRAM (Concurrent Read Exclusive Write Parallel RAM - a theoretical construct simulating a type of computer), it takes at least $s(f)$ steps to compute the value of $f$. Nisan \cite{N} later introduced the concept of block sensitivity and observed that $s(f)$ can be replaced by $bs(f)$ for the lower bound of CREW PRAM's 
	computation time.  
	Nisan and Szegedy \cite{NS} proved that block sensitivity is polynomially related to other complexity measures such as decision tree complexity, certificate complexity, and degree of the polynomial representation of a Boolean function.  For more information on such polynomial relations between complexity measures, we refer the reader to \cite{HKP}.  The surprising polynomial relations between almost all types of complexity measures of Boolean functions motivate the search for an answer to the Sensitivity Conjecture. 
	


	H.-U. Simon \cite{S} proved that for any Boolean function $f$, we have $s(f) \geq 1/2\log n -1/2\log{\log n} + 1/2$, where $n$ is the number of effective variables, that is, the number of variables on which the function actually depends. This lower bound is tight up to the additive  $O(\log{\log n})$. This gives us an exponential upper bound on the block sensitivity in terms of sensitivity. Kenyon and Kutin \cite{KK} introduced the notion of $\ell$-block sensitivity and gave a better upper bound, $O(\frac{e}{\sqrt{2\pi}}e^{s(f)}\sqrt{s(f)})$, which is still exponential in sensitivity.

	A function $f$ is called monotone if $f(x) \leq f(y)$ whenever $y$ can be obtained by flipping the 0's of $x$ to 1's. Nisan \cite{N} proved that for monotone Boolean functions, sensitivity and block sensitivity are asymptotically the same.  

	For graph properties, Turan \cite{T} proved if $f$ is a graph property on $v$ vertices, then $s(f) = \Omega(v) = \Omega(\sqrt{n})$, which implies that the gap is at most quadratic for graph properties. In the same paper, he provides an example, the ``isolated vertex'' property, which shows that the lower bound is tight.

	\begin{exmp} ({\it Isolated vertex property}) 
		Let $f$ be the graph property on $v$ vertices such that $f(G) = 1$ if and only if $G$ has an isolated vertex. The sensitivity of $f$ is $v-1$. Note that this property is monotone, so $bs(f) = s(f)$.
	\end{exmp}
			
\section{$k$-Uniform  Hypergraph Property with $\Theta(v^{k/2})$ Sensitivity}
	The isolated vertex property in the previous section gives a graph property with sensitivity exactly $\Theta(v)$. In this section, we show $k$-uniform graph properties with $\Theta(v^{k/2})$ sensitivity for all $k \geq 3$.

	\begin{theorem}
		Given any $k,i \in \mathbb{N}$ with $k\geq 3$, $1 \leq i < k$ and any $t \in [0,1]$, there exists a $k$-uniform hypergraph property $f$ such that $s(f) = max\{O(v^{i(1-t)}), \Theta(v^{k-i(1-t)})\}$.
	\end{theorem}
	\begin{proof}
		Let $\mathcal{H}$ be a $k$-uniform clique on $\Theta(v^t)$ vertices.  Define a $k$-uniform hypergraph property $f$ such that $f=1$ if there exists a copy of $\mathcal{H}$ inside the graph such that given any edge $E$ not entirely in $\mathcal{H}$, we have
		\begin{equation}
			|V(\mathcal{H}) \cap E| < i. \nonumber
		\end{equation}
		\indent We claim that this is the desired $k$-uniform hypergraph property. First we consider $s^0(f)$. When there's no desired $\mathcal{H}$ inside the graph, we call $v^t$ vertices $\{w_1,...w_{v^t}\}$ a sensitive tuple if we can form a desired $\mathcal{H}$ by either removing or adding one edge. 
		\begin{lem}
			Every sensitive tuple contains exactly one sensitive edge.
		\end{lem}
		\begin{proof}
			If the tuple can form a desired $\mathcal{H}$ by removing one edge, then it is a desired clique but does not fit the isolation criterion. There can only be one edge that needs to be removed, as if there were two edges that violated the isolation criterion, then by removing only one of them the value of the function wouldn't change and so it wouldn't be a sensitive tuple.
If the tuple can form a desired $\mathcal{H}$ by adding one edge, then it satisfied the isolation criterion and is one edge short of being a clique. But then there's only one edge that can be added to those points, as every other edge in the clique is already existent, by the definitions of clique and our isolation property.
		\end{proof}
		 Then every sensitive tuple contains exactly one sensitive edge. Let $\mathcal{F}$ denote the set of all sensitive tuples.  We have
		\begin{equation}
			s^0(f) \leq |\mathcal{F}|. \nonumber
		\end{equation}
		Then we can get an upper bound of $s^0(f)$ by finding an upper bound of $|\mathcal{F}|$. We claim that no two elements of $\mathcal{F}$ can share more than $i-1$ vertices. Suppose we have $A=\{w_1,...,w_{j}\}$ and $B=\{w_1^\prime,...,w_{j}^\prime\}$ two sensitive tuples such that they have $i$ common vertices $\{v_1,...,v_i\}$ and $i<j$, because otherwise they are the same tuple (here $j$ is $\Theta(v^t)$).  Since $j$ grows with $v$ and $k$ is constant, we can consider larger graphs with $v^{t} > k + 2$.  Using this assumption, if both of our tuples can form a desired $\mathcal{H}$ by adding one edge, they are both almost cliques - they each have every edge required except one.  Then there is already at least one edge $E_{1}$ such that $\{v_1,..,v_i\} \subset E_{1} \subset A$ and a different edge $E_{2}$ satisfying $\{v_1,..,v_i\} \subset E_{2} \subset B$, 
		because in both $A$ and $B$ every subset of size $k$ is present as an edge except one, and there are more than one such subsets containing $\{v_1,...,v_i\}$ because 
		$j > k+2 > i+2$.  
		Then adding one edge to complete $A$ would still leave $E_{2}$, which is not contained in $A$ and violates our isolation condition, so $A$ would not actually be a 
		desired isolated subgraph.  Similarly adding an edge to $B$ cannot actually produce an isolated clique as desired, because of the edge(s) leaking into $A$.  The proof for 
		even more than $i$ vertices in common is the same - as long as $|A \cap B|<|A| - 1$, there are too many edges not entirely in $A$ or $B$ that violate our isolation condition. \\
		\indent
		If one of these two tuples can form a desired $\mathcal{H}$ by adding one edge and the other by removing an edge, then either $A$ or $B$ is already a clique with one extra edge violating the isolation condition, so similarly any attempt to make $A$ or $B$ into the desired $\mathcal{H}$ will not be possible because of overlapping edges containing points in
		$\{v_1,...,v_i\}$, which are required by the definition of clique.  \\
		 \indent
		If both tuples can form $\mathcal{H}$ by removing one edge, again there are at least $j-i \choose{k-1}$ edges containing vertices in $\{v_1,...,v_i\}$ that are overlapping 
		into our desired cliques, and with enough vertices this is far too many.  \\
		 \indent Thus we know that any set of $i$ vertices is contained in no more than one sensitive tuple in $\mathcal{F}$ and any sensitive tuple contains $v^t \choose{i}$ distinct sets of $i$ vertices. So we have
		\begin{equation}
			s^0(f) \leq |\mathcal{F}| \leq \frac{{v \choose{i}}}{{v^t \choose{i}}} = O(v^{i(1-t)}). \nonumber
		\end{equation}
		Finally we consider $s^1(f)$. When there exists such an $\mathcal{H}$ inside the graph, the only way to eliminate it is to either remove an edge inside $\mathcal{H}$ or add an edge with more than $i-1$ vertices in $V(\mathcal{H})$. Thus we have
		\begin{equation}
			s^1(f) = {v^t \choose{k}} + {v^t \choose{i}}{v-i \choose{k-i}} = \Theta (v^{k-i(1-t)}). \nonumber
		\end{equation}
		The second term represents adding an edge with too many vertices in $V(\mathcal{H})$ - we only need to consider the case of having exactly 
		$i$ vertices in $\mathcal{H}$ because it is asymptotically the dominating term.  It then follows that we have
		\begin{equation}
			s(f)= \max\{s^0(f),s^1(f)\} = \max\{ O(v^{i(1-t)}),  \Theta(v^{k-i(1-t)})\}. \nonumber
		\end{equation}
	\end{proof}
	By looking at special cases of this result, we can find a $k$-uniform hypergraph property for any $k$ odd with sensitivity $O(v^{k/2})$.
	\begin{cor}
		There exists a $k$-uniform hypergraph property with sensitivity $O(v^{k/2})$, for all $k$.
	\end{cor}
	\begin{proof}
		An example with $k=2$ was given before. For $k\geq 3$, any solution to the equation $i(1-t)=\frac{k}{2}$ for $t\in[0,1]$ and $i\in\mathbb{N}$ works, and specifically for every $i\geq\frac{k}{2}$, setting $t=1-\frac{k}{2i}$ yields the desired result.
	\end{proof}

	We now introduce a lemma which will allow us, in certain cases, to obtain a tight lower bound on $s^0(f)$.

	\begin{lem}
		Let $q$ be a prime power and $d \leq q$. $\forall\ell\in\mathbb{N}$, there exists a collection of sets $S_1, ..., S_m \subseteq [q^{\ell+1}]$ such that $|S_i| = q$ for all $i$ and $|S_i \cap S_j| < d$ for $i \neq j$ and $m = q^{d\ell}$.
	\end{lem}
	\begin{proof}
		Let $f : \mathbb{F}_q \rightarrow \mathbb{F}^\ell_q$ be such that $f(x) = (f_1(x), ... ,f_k(x))$ where each $f_i$ is a degree $k-1$ polynomial over $\mathbb{F}_q$. Then each $f$ corresponds to a set of $q$ $(k+1)$-tuples, $S_f = \{(x,f_1(x),...,f_k(x))\text{ }  |\text{ } x \in \mathbb{F}_q\}$. 

		If $g \neq f$, then the sets $S_f$ and $S_g$ intersect in at most $k-1$ points since the equation $f_1(x) = g_1(x)$ already has at most $k-1$ solutions in $\mathbb{F}_q$. We can relabel $S_f$ by associating $i \in [q^{k+1}]$ to each $f : \mathbb{F}_q \rightarrow \mathbb{F}^k_q$. There are $q^k$ distinct polynomials over $\mathbb{F}_q$ of degree $k-1$, so there are $(q^k)^k$ distinct sets of $(k+1)$-tuples that satisfy the above property. Hence, we can construct a collection of sets $S_1, ..., S_{q^{k^2}}$ such that $|S_i| = q$ for all $i$ and $|S_i \cap S_j| < k$ for $i \neq j$.
	\end{proof}
	\begin{prop}
		The upper bound in the Corollary 3.5 is tight, that is, $\forall k \in \mathbb{N}, \exists i\in\mathbb{N},t\in[0,1]$ such that if $f$ is the function in Corollary 3.5 with parameters
	$t$, $i$, and $k$, then $s(f)=\Theta(v^\frac{k}{2})$
	\end{prop}
	\begin{proof}
		It is sufficient to prove that, in a special case of Theorem 3.3, the upper bound on $s^0(f)$ is tight, as we have already shown $s^{1}(f)$ is tight. We will prove this by evoking the lemma just introduced.

		First, consider the case of $k$ even. $i=\frac{k}{2},t=1$ is an assignment yielding $s(f)=O(v^\frac{k}{2})$, which we will prove to be tight.  Let $q$ be a prime power such that $k+1<q<2(k+1)$, the existence of which is guaranteed by Bertrand's Postulate. Let $\ell=\lfloor\log_q(v)-1\rfloor\geq\log_q(v)-2$, and $d=\frac{k}{2}$. Then by the lemma we have $m=q^{d\ell}=\Omega(v^\frac{k}{2}q^{-2\frac{k}{2}})=\Omega(v^\frac{k}{2}q^{-k})=\Omega(v^\frac{k}{2})$ many sets $S_1,\ldots S_m$ such that $|S_i|\geq k+1$ and $\forall i\neq j , |S_i\cap S_j|<\frac{k}{2}$. Let $S'_i$ be any subset of $S_i$ such that $|S'_i|=k+1$. Note that $\forall i\neq j,|S'_i\cap S'_j|<\frac{k}{2}$.

		Now we can construct an input for which $s^0(f)=\Omega(v^\frac{k}{2})$. Label the vertices $1,\ldots ,v$ and partition a subset of these into the sets $S'_i$. Make the sets of points $S'_i$ be cliques, and then remove any one edge from each of the cliques. Since no two share $\frac{k}{2}$ points, the isolation criterion is automatically satisfied, so re-adding the removed edge to any of the almost-cliques changes the value of $f$ from $0$ to $1$. There are $\Omega(v^\frac{k}{2})$ many sets $S'_i$, so $s^0(f)=\Omega(v^\frac{k}{2})$.

		Next, for the case of $k$ odd.  $i=\frac{k+1}{2},t=\frac{1}{k+1}$ is an assignment yielding\\
$s(f)=O(v^\frac{k}{2})$, which we will prove to be tight. Let $q$ be a prime power such that $\frac{1}{2}v^t<q<v^t$, the existence of which is again guarenteed by Bertrand's Postulate.  Let $\ell=\frac{1}{t}-1=k$, and $d=\frac{k+1}{2}$. Then by the previous lemma, we have that $m=q^{d\ell}=\Omega(v^{td\ell})=\Omega(v^{\frac{1}{k+1}\frac{k+1}{2}k})=\Omega(v^\frac{k}{2})$ many sets $S_1,\ldots S_m$ such that $|S_i|\geq k+1$ such that $\forall i\neq j,|S_i\cap S_j|<\frac{k}{2}$. We proceed to construct our input as before, and again obtain $s^0(f)=\Omega(v^\frac{k}{2})$.
	\end{proof}

	\begin{rem}
		Although it might appear that Lemma 3.4 could be used to obtain a general tight lower bound on $s^0(f)$, there is an important caveat. Since we consider $\mathbb{F}^\ell_q$, $\ell$ must be a positive integer. However, using our argument, except in the case of $t=0$, $l$ is an integer if and only if $\frac{1}{t}$ is, and since $t\in[0,1]$, this is not guaranteed in general.
	\end{rem}

\section{Gaps in Sensitivity and Block Sensitivity for Graph Properties}
	In this section, we give a $k$-uniform hypergraph property that demonstrates a quadratic gap between sensitivity and block sensitivity for every even $k$. We first start with a graph property.

	\begin{theorem}
		There exists a graph property $f$ such that $s(f) = \Theta(v)$ and $bs(f) = \Theta(v^2)$.
	\end{theorem} 

	\begin{proof}
		Let $f$ be the graph property, ``there exists an isolated triangle.''  By isolated triangle, we mean three vertices $a,b,c$ with edges $\{a,b\}, \{a,c\}, \{b,c\}$ and no 
		edges between any one of $a$,$b$, or $c$ and any $v \in V\setminus\{a,b,c\}$.

		Consider the block sensitivity on an empty graph input. A triangle is not edge-disjoint with $3(v-3)$ other triangles and there are ${v \choose 3}$ possible triangles on a graph on $v$ vertices. It follows that there are at least $ \frac{{v \choose 3}}{3(v-3)} = \Omega(v^2)$ edge-disjoint triangles on $v$ vertices. Hence, $bs(f) = \Omega(v^2)$ because we can just group the edges as these disjoint triangles. Since $bs(f) \leq {v \choose 2}$ because there are only this many vertices, it follows that $bs(f) = \Theta(v^2)$.

		For sensitivity, we consider $s^0(f)$ first. To change from 0 to 1, we need to either add an edge to complete an isolated triangle or remove an edge so that a triangle becomes isolated. We call a triple $(v_1,v_2,v_3)$ sensitive if adding or removing an edge from the graph will make the $(v_1,v_2,v_3)$ vertices an isolated triangle.

	\begin{claim}
		Sensitive triples are pairwise vertex disjoint. 
	\end{claim}
	\begin{proof}
		To see this, let two distinct sensitive triples be $(a,v_1,v_2)$ and $(a,w_1,w_2)$. Since $(a,v_1,v_2)$ is 1 bit away from being an isolated triangle, $a$ cannot share an edge with both $w_1$ and $w_2$. In fact, $a$ shares an edge with exactly one of them because otherwise $(a,w_1,w_2)$ is not 1 bit from being an isolated triangle. By the same logic, $a$ shares an edge with exactly one of $v_1, v_2$. This implies that induced subgraphs on $(a,v_1,v_2)$ and $(a,w_1,w_2)$ are trees. However, this means that $(a,v_1,v_2)$ cannot be sensitive since there is an edge connecting $a$ to a vertex other than $v_1,v_2$. Hence, sensitive triples cannot share a vertex. 
	\end{proof}

	Now we can prove that $s^0(f) = \Theta(v)$. Let $C_1, C_2, ... C_m$ be sensitive 3-tuples on a graph $G$. Since two sensitive 3-tuples cannot share a vertex, a vertex uniquely determines a sensitive 3-tuple. Hence, $m \leq v$. Since we can have $v/3$ vertex disjoint trees consisting of 3 vertices on $G$, $s^0(f) = \Theta(v)$.

	Next we consider $s^1(f)$. To change from 1 to 0, we need to either remove an edge from the isolated triangle or add an edge to it so that it is no longer isolated. For the first case, it is easy to see that 3 bits are sensitive. For the second case, at most $3(v-3)$ bits can make the triangle not isolated, so $s^1(f) = \Theta(v),$ and therefore $s(f) = \max\{\Theta(v),\Theta(v)\}=\Theta(v)$.
	\end{proof}


	Then we can generalize this graph property to a $k$-uniform hypergraph property for any $k \in \mathbb{N}$, which gives a quadratic gap for $k$ even and nearly a quadratic gap for $k$ odd.
	\begin{theorem}
		Let $k,i \in \mathbb{N}$ and $1 \leq i \leq k/2$.  There exists a $k$-uniform hypergraph property $f$ on $v$ vertices such that $s(f) = \Theta(v^{k-i})$ and $bs(f) = \Theta(v^{k})$.
	\end{theorem}
	\begin{proof}
		Consider the function from Theorem 3.1, with $t$ set to 0. Since $i\leq\frac{k}{2}$, we have that $i\leq k-i$, so $s(f)=\Theta(v^{k-i})$. We now wish to show that $bs(f)=\Theta(v^k)$.
		\begin{lem}
			In a $k$-uniform hypergraph on $v$ vertices, there are $\Omega (v^k)$ edge disjoint $K_{k+1}^{(k)}$.
		\end{lem}
		\begin{proof}
			Note that $t=0$ because we are considering cliques of a constant size, $k+1$.  There are $v \choose{k+1}$  many ($k+1$)-cliques, and for every clique chosen, we eliminate any other cliques with more than $k-1$ points in common. In this way we get at least
		\begin{equation}
			\frac{{v \choose{k+1}}}{{k+1 \choose{k}}(v-k)} = c*v^{k} \nonumber
		\end{equation}		 
		many cliques, and any two of them have no more than $k-1$ points in common, which guarantees that they are edge-disjoint cliques (no edge is in two different ones).
		\end{proof}
		Consider the empty graph. By the lemma, there are $\Omega (v^k)$ disjoint cliques and each of them is a sensitive block. Hence we have $bs(f) = \Theta (v^k)$ by applying the trivial upper bound on block sensitivity.
	\end{proof}
	
	\begin{cor}
		For all $k \in \mathbb{N}$, there exists a $k$-uniform hypergraph property on $v$ vertices such that $s(f) = \Theta(v^{\lceil k/2 \rceil})$ and $bs(f)= \Theta (v^k)$.
	\end{cor}
	\begin{proof}
		When $k$ is even, choose $i=k/2$ and when $k$ is odd, choose $i=(k-1)/2$ in above proposition.
	\end{proof}
	
	\begin{rem}
		This proof can be generalized past the specific subgraph of  $K_{k+1}^{(k)}$, but to what extent, we are not certain.  Two possible types of subgraphs for which the proof could apply are ``near-cliques,'' or subgraphs that are a constant number of edges away from being a clique, and subgraphs in which every vertex has different degree.  The idea is to preserve the fact that there is only one way to complete a sensitive tuple by adding a single edge.  Certain types of highly symmetric subgraphs do not fit this criterion - for example, a simple cycle of ordered vertices $\{v_1, \ldots , v_w\}$ with one ``chord'' edge added, say $\{v_1,v_{w/2}\}$, can be completed (up to isomorphism) from a simple cycle in a number of ways that is $\Theta(w)$, where the cycle has $w$ vertices.  
	\end{rem}
	
	\begin{rem}
		Also, for $k$ even, our $k$-uniform hypergraph property has sensitivity $O(v^{k/2})$.  For $k$ odd, the sensitivity is $O(v^{(k+1)/2})$, an upper bound strictly greater than $O(v^{k/2})$.
	\end{rem}

\begin{prop}
		Let $f$ be the property defined in Corollary 3.3. Then $s(f) = O(v^{3/2})$ and $bs(f) = \Omega(v^{9/4})$.
	\end{prop}
	
	\begin{proof}
		For the block sensitivity, we consider the block sensitivity on the empty graph. The block sensitivity on the empty graph is equal to the maximum number of edge disjoint cliques of size $v^{1/4}$ on $v$ vertices. Assuming that $v$ is a prime power, we can use Lemma 4.1 and conclude that $bs(f) \geq v^{9/4}$.
	\end{proof}
	\begin{conj}
		Let $f$ be the property defined in above proposition. $bs(f) = \Theta(v^{9/4})$.
	\end{conj}
	
	\section{Further Work and Open Problems}
		Although we have fully answered Conjecture 1, significant improvements can still be made. It remains unknown if Conjecture 2 holds for the case of $k$ odd, and we have made no progress on Conjecture 3 at all, and our approach to Conjectures 1 and 2 does not seem relevant to Conjecture 3. Of lesser import is to finish the work we began in Section 4, and analyze the block sensitivity for the function defined in Corollary 3.3.

		The most promising continuation of this work appears to be the following slight modification to the functions presented in this paper: One way of viewing the function in Theorem 3.1 is as the indicator function of the existence of a certain subgraph with a certain degree of isolation. Although in this paper we restrict to the view of when that subgraph is a clique, there are other subgraphs that we can look for to obtain the same result. For example, if the subgraph has the property that every vertex has different degree, then a slight modification to the proof yields the same result. It seems quite likely however, that a more cleverly chosen subgraph might yield better results, especially in the case of $k$ odd. Although it would follow from a proof of Conjecture 3 that our function achieves the lowest sensitivity possible for even $k$, there might be a function that achieves the same sensitivity for odd k as well by choosing a different subgraph to look for.
	
	In summary, we still have the following statements open:
	\begin{conj*}
		There exists, for every $k$ odd, a $k$-uniform hypergraph property $f$ such that $s(f)$ and $bs(f)$ display a quadratic gap. That is,  $bs(f)=\Omega(s(f)^2)$.
	\end{conj*}
	
	\begin{conj*}
		For every $k$-uniform hypergraph property $f,$ $$s(f)=\Omega(\sqrt{n})=\Omega(v^{k/2}).$$
	\end{conj*}

	\subsection*{Acknowledgements}
		We would like to thank our mentor, Professor Laci Babai, for introducing us to the Sensitivity Conjecture and its related problems, and providing us with many helpful comments.  Without his guidance and suggestions, this paper would not have been possible.  We also thank Professor Babai and Professor Stuart Kurtz for organizing the wonderful Computer Science REU during which this paper was completed.  

\begin{thebibliography}{6}

\bibitem{C}
Sourav Chakraborty.
{\it On the sensitivity of cyclically-invariant Boolean functions}.
Proc. 20th Ann. IEEE Conf. Comput. Complexity (CCC), pp. 163-167.
IEEE Comp. Soc. Press, 2005.

\bibitem{CDR}
Stephen Cook, Cynthia Dwork, and Rudiger Reishuk
{\it Upper and lower time bounds for parallel random access machines without simultaneous writes}
SIAM J.Comput. 15 (1986), no. 1, 87-97

\bibitem{HKP}
Pooya Hatami, Raghav Kulkarni, Denis Pankratov
{\it Variations on the Sensitivity Conjecture}
Theory of Computing Graduate Surveys 4 (2011), pp. 1-27

\bibitem{KK}
Claire Kenyon, Samuel Kutin.
{\it Sensitivity, block sensitivity, and $l$-block sensitivity of Boolean functions}.
Information and Computation, vol 189 (2004), no. 1, 43-53.

\bibitem{N}
Noam Nisan.
{\it CREW PRAMS and decision trees}.
SIAM J. Comput. 20 (1991), no. 6, 999-1070

\bibitem{NS}
Noam Nisan, Mario Szegedy.
{\it On the degree of Boolean functions as real polynomials.}
Comput. Complexity, 4:462-467, 1992.

\bibitem{R}
David Rubinstein.
{\it Sensitivity vs. block sensitivity of Boolean functions}.
Combinatorica 15 (1995), no. 2, 297-299.

\bibitem{S}
Hans-Ulrich Simon.
{\it A tight $\Omega(\log{\log{n}})$-bound on the time for parallel RAM's to compute nondegenerated Boolean functions}. FCT vol 4 (1983), Lecture notes in Comp. Sci. 158.

\bibitem{T}
Gyorgy Turan.
{\it The critical complexity of graph properties}.
Inform. Process. Lett. 18 (1984). 151-153.

\end{thebibliography}

\end{document}