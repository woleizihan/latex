\documentclass[psamsfonts]{amsart}

%-------Packages---------
\usepackage{amssymb,amsfonts}
\usepackage[all,arc]{xy}
\usepackage{enumerate}
\usepackage{mathrsfs}

%--------Theorem Environments--------
%theoremstyle{plain} --- default
\newtheorem{theorem}{Theorem}[section]
\newtheorem{cor}[theorem]{Corollary}
\newtheorem{prop}[theorem]{Proposition}
\newtheorem{lem}[theorem]{Lemma}
\newtheorem{conj}[theorem]{Conjecture}
\newtheorem{quest}[theorem]{Question}
\newtheorem{obs}[theorem]{Observation}

\theoremstyle{definition}
\newtheorem{defn}[theorem]{Definition}
\newtheorem{defns}[theorem]{Definitions}
\newtheorem{con}[theorem]{Construction}
\newtheorem{exmp}[theorem]{Example}
\newtheorem{exmps}[theorem]{Examples}
\newtheorem{notn}[theorem]{Notation}
\newtheorem{notns}[theorem]{Notations}
\newtheorem{addm}[theorem]{Addendum}
\newtheorem{exer}[theorem]{Exercise}

\theoremstyle{remark}
\newtheorem{rem}[theorem]{Remark}
\newtheorem{rems}[theorem]{Remarks}
\newtheorem{warn}[theorem]{Warning}
\newtheorem{sch}[theorem]{Scholium}

\makeatletter
\let\c@equation\c@theorem
\makeatother
\numberwithin{equation}{section}

\bibliographystyle{plain}
%\;\;\makebox[0pt]{$\top$}\makebox[0pt]{$\cap$}\;\

\begin{document}
	\begin{lem}
	Let $q$ be a prime power, and $d \leq  q-1$. Then there exists a collection of sets $S_1, ..., S_m \subseteq [q^{l+1}]$ such that $|S_i| = q$ for all $i$ and $|S_i \cap S_j| < d$ for $i \neq j$ and $m = q^{dl}$.
	\end{lem}
	\begin{proof}
	Let $f : \mathbb{F}_q \rightarrow \mathbb{F}^l_q$ such that $f(x) = (f_1(x), ... ,f_k(x))$ where each $f_i$ is a degree $d-1$ polynomial over $\mathbb{F}_q$. Then each $f$ corresponds to a set of $l+1$-tuples, $S_f = \{(x,f_1(x),...,f_k(x))\text{ }  |\text{ } x \in \mathbb{F}_q\}$. 

If $g \neq f$, then the sets $S_f$ and $S_g$ intersect at most $d-1$ points since the equation $f_1(x) = g_1(x)$ already has at most $d-1$ solutions in $x \in \mathbb{F}_q$. We can relabel $S_f$ by associating $i \in [q^{l+1}]$ to each $f : \mathbb{F}_q \rightarrow \mathbb{F}^l_q$. There are $q^d$ distinct polynomials over $\mathbb{F}_q$ of degree $d-1$, so there are $(q^d)^l$ distinct sets of $(l+1)$-tuples that satisfy the above property. Hence, we can construct a collection of sets $S_1, ..., S_{q^{dl}}$ such that $|S_i| = q$ for all $i$ and $|S_i \cap S_j| < d$ for $i \neq j$.
	\end{proof}
	\begin{cor}
		The upper bounds on 0-sensitivity of our graph properties are tight.
	\end{cor}
	\begin{proof}
		For $s^0(f)$ of our first graph property (Theorem 3.2), let $q$ be the prime power between $k+1$ and $2(k+1)$, $l = \lfloor \log_q(v)-1 \rfloor \geq \log_q(v)-2$ and $d=i$. Then we have $m = q^{dl} \geq \Omega(v^iq^{-2i}) = \Omega(v^i)$ many sets $S_1,....,S_m$ such that $S_i \cap S_j < d$.\\
		\indent For second graph property (Theorem 4.2), let $q$ be the prime power between $0.5v^t$ and $v^t$, $l = 1/t - 1$. Then we have $m = q^{dl} \geq \Omega(v^iq^{-2i}) = \Omega(v^{i(1-t)})$ many sets $S_1,....,S_m$ such that $S_i \cap S_j < d$.\\
	\end{proof}
\end{document}