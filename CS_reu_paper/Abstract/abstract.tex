\documentclass[psamsfonts]{amsart}

%-------Packages---------
\usepackage{amssymb,amsfonts}
\usepackage[all,arc]{xy}
\usepackage{enumerate}
\usepackage{mathrsfs}

%--------Theorem Environments--------
%theoremstyle{plain} --- default
\newtheorem{thm}{Theorem}[section]
\newtheorem{cor}[thm]{Corollary}
\newtheorem{prop}[thm]{Proposition}
\newtheorem{lem}[thm]{Lemma}
\newtheorem{conj}[thm]{Conjecture}
\newtheorem{quest}[thm]{Question}

\theoremstyle{definition}
\newtheorem{defn}[thm]{Definition}
\newtheorem{defns}[thm]{Definitions}
\newtheorem{con}[thm]{Construction}
\newtheorem{exmp}[thm]{Example}
\newtheorem{exmps}[thm]{Examples}
\newtheorem{notn}[thm]{Notation}
\newtheorem{notns}[thm]{Notations}
\newtheorem{addm}[thm]{Addendum}
\newtheorem{exer}[thm]{Exercise}

\theoremstyle{remark}
\newtheorem{rem}[thm]{Remark}
\newtheorem{rems}[thm]{Remarks}
\newtheorem{warn}[thm]{Warning}
\newtheorem{sch}[thm]{Scholium}


\makeatletter
\let\c@equation\c@thm
\makeatother
\numberwithin{equation}{section}

\bibliographystyle{plain}

%--------Meta Data: Fill in your info------
\title{Abstract}

\begin{document}
\maketitle

Many different complexity measures exist for Boolean functions besides sensitivity and block sensitivity.  However, it has been shown [Hatami, Kulkarni, Pankratov] that block sensitivity 
is polynomially related to several other important complexity measures, including certificate complexity, decision-tree complexity, and degree as a multilinear polynomial.  
Two complexity measures $A(f)$ and $B(f)$ are said to be \textbf{polynomially related} if $\exists$ polynomials $ p_{1}(x),  \: p_{2}(x) $ such that $\forall$ Boolean functions $f$, 
$A(f) \leq p_{1}(B(f))$ and $B(f) \leq p_{2}(A(f))$.  
If two complexity measures are polynomially related, they are viewed in some sense as measuring the same quantity about Boolean functions, and we regard them as interchangeable.  
It is clear from the definitions that $\forall$ Boolean functions $f$, $s(f) \leq bs(f)$.  A polynomial bound in the other direction has not yet been proven and is known as the 
\textbf{sensitivity conjecture}.  In fact, a stronger conjecture by Nisan and Szegedy is that $bs(f) \leq s(f)^{2}$.  Boolean functions $f$ satisfying $bs(f) = \Omega(s(f)^{2})$ are said to 
demonstrate a \textbf{quadratic gap} between block sensitivity and sensitivity.  In this paper we exhibit a class of Boolean functions achieving this 
quadratic gap.  Our functions are all $k$-graph properties, or Boolean functions that involve decoding a binary string to create a 
$k$-uniform hypergraph
and determining whether that graph has some property.  $k$-graph properties are Boolean functions on input strings of size ${v \choose k}$, with each bit representing the presence or 
absence of an edge.  Graph properties are invariant under graph isomorhism and thus under relabeling of the vertices and corresponding different encodings of the graph.  Specifically, we present a class of functions demonstrating a quadratic gap for all $k$ even and a nearly-quadratic gap for $k$ odd.  We also present a graph property with 
sensitivity $O(v^{\frac{k}{2}})=O(\sqrt n)$ for all $k$.  
Finally, we include a general criterion for determining whether a certain type of graph property will achieve a quadratic gap.  


\end{document}