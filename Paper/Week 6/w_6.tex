\documentclass[psamsfonts]{amsart}

%-------Packages---------
\usepackage{amssymb,amsfonts}
\usepackage[all,arc]{xy}
\usepackage{enumerate}
\usepackage{mathrsfs}

%--------Theorem Environments--------
%theoremstyle{plain} --- default
\newtheorem{theorem}{Theorem}[section]
\newtheorem{cor}[theorem]{Corollary}
\newtheorem{prop}[theorem]{Proposition}
\newtheorem{lem}[theorem]{Lemma}
\newtheorem{conj}[theorem]{Conjecture}
\newtheorem{quest}[theorem]{Question}
\newtheorem{obs}[theorem]{Observation}

\theoremstyle{definition}
\newtheorem{defn}[theorem]{Definition}
\newtheorem{defns}[theorem]{Definitions}
\newtheorem{con}[theorem]{Construction}
\newtheorem{exmp}[theorem]{Example}
\newtheorem{exmps}[theorem]{Examples}
\newtheorem{notn}[theorem]{Notation}
\newtheorem{notns}[theorem]{Notations}
\newtheorem{addm}[theorem]{Addendum}
\newtheorem{exer}[theorem]{Exercise}

\theoremstyle{remark}
\newtheorem{rem}[theorem]{Remark}
\newtheorem{rems}[theorem]{Remarks}
\newtheorem{warn}[theorem]{Warning}
\newtheorem{sch}[theorem]{Scholium}

\makeatletter
\let\c@equation\c@theorem
\makeatother
\numberwithin{equation}{section}

\bibliographystyle{plain}
%\;\;\makebox[0pt]{$\top$}\makebox[0pt]{$\cap$}\;\

\begin{document}
	\section{The Duality Theorem}
		\begin{defn}
			For an arbitrary space $X$ and coefficitn ring $R$, define an
			R-linear cap product $\frown : C_k(X;R)\times C^l(X;R) \to C_{k-l}
			(X;R)$ for $k \geq l$ by setting
				\begin{equation}
					\sigma \frown \phi = \phi(\sigma | [v_0,...,v_l]) \sigma | [v_l,...,v_k]
				\end{equation}
				for $\sigma \Delta^k \to X$ and $\phi \in C^k(X;R)$.
		\end{defn}
		This induces a cap product in homology and cohomology by the formula
		\begin{equation}
			\partial (\sigma \frown \phi) = (-1)^l (\partial \sigma \frown \phi - \sigma \from \delta \phi).
		\end{equation}
		\begin{theorem}
			If $M$ is a closed $R$-orientable $n$-manifold with fundamental class
			$[M] \in H_n(M;R)$, then the map $D: H^k(M;R) \to H_{n-k}(M;R)$ defined 
			by $D(\alpha) = [M]\frown \alpha$ is an isomorphism for all $k$.
		\end{theorem}
\end{document}