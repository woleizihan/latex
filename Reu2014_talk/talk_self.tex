\documentclass[psamsfonts]{amsart}

%-------Packages---------
\usepackage{amssymb,amsfonts}
\usepackage[all,arc]{xy}
\usepackage{enumerate}
\usepackage{mathrsfs}

%--------Theorem Environments--------
%theoremstyle{plain} --- default
\newtheorem{theorem}{Theorem}[section]
\newtheorem{cor}[theorem]{Corollary}
\newtheorem{prop}[theorem]{Proposition}
\newtheorem{lem}[theorem]{Lemma}
\newtheorem{conj}[theorem]{Conjecture}
\newtheorem{quest}[theorem]{Question}
\newtheorem{obs}[theorem]{Observation}

\theoremstyle{definition}
\newtheorem{defn}[theorem]{Definition}
\newtheorem{defns}[theorem]{Definitions}
\newtheorem{con}[theorem]{Construction}
\newtheorem{exmp}[theorem]{Example}
\newtheorem{exmps}[theorem]{Examples}
\newtheorem{notn}[theorem]{Notation}
\newtheorem{notns}[theorem]{Notations}
\newtheorem{addm}[theorem]{Addendum}
\newtheorem{exer}[theorem]{Exercise}

\theoremstyle{remark}
\newtheorem{rem}[theorem]{Remark}
\newtheorem{rems}[theorem]{Remarks}
\newtheorem{warn}[theorem]{Warning}
\newtheorem{sch}[theorem]{Scholium}

\makeatletter
\let\c@equation\c@theorem
\makeatother
\numberwithin{equation}{section}

\bibliographystyle{plain}
%\;\;\makebox[0pt]{$\top$}\makebox[0pt]{$\cap$}\;\

\begin{document}
	\section{Lefschetz fixed point theorem}
	Let's first start with the definitions and basic properties of Lefschetz number. I'll assume some intersection theory, like what's the intersection number and theorems like any map is homotopic to a map that intersect transversally with a manifold. 
	\begin{defn}
			The global Lefschetz number of $f$ is the intersection number $I(\Delta, graph(f))$, denoted $L(f)$.
		\end{defn}
		Then we know that given any map without a fixed point, it's Lefschetz number is $0$, which gives the following theorem.
		\begin{theorem}(Smooth Lefschetz Fixed-Point Theorem)
			Let $f:X \to X$ be a smooth map on a compact orientable manifold. If $L(f) \neq 0$, then $f$ has a fixed point.			
		\end{theorem}
		Since Lefschetz number is defined as a special intersection number	and intersection number is homotopy invariant, we know that Lefschetz number is also homotopy invariant.
		\begin{prop}
			$L(f)$ is a homotopy invariant.
		\end{prop}
		Euler characteristic is also defined as the intersection number $I(\Delta, \Delta)$. Then we know:
		\begin{prop}
			If $f$ is homotopic to the identity,  $L(f)$ equals the Euler characteristic of $X$. In particular, if $X$ admits a smooth map $f: X \to X$ that is homotopic to the identity and has no fixed points, then $\chi(X) = 0$.
		\end{prop}
		Now we look at a subcollection of "nice" functions, whose graph intersect diagonal transversally.
		\begin{defn}
			$f: X \to X$ is a Lefschetz map if $graph(f)$ \;\;\makebox[0pt]{$\top$}\makebox[0pt]{$\cap$}\;\ $\Delta$
		\end{defn}
		Then by some intersection theory, we know that  
		\begin{prop}
			Every map $f: X \to X$ is homotopic to a Lefschetz map.
		\end{prop}
		And thus looking at Lefschetz maps is enough since Lefschetz number is homotopy invariant\\
		\indent Given any $x$ a fixed point of a  map, we have $graph(f)$ \;\;\makebox[0pt]{$\top$}\makebox[0pt]{$\cap$}\;\ $\Delta$ if and only if
		\begin{equation}
			graph(df_x) + \Delta_x = T_x(X) \times T_x(X).
		\end{equation}
		And this implies that $df_x$ has no nonzero fixed point.
		\begin{defn}
			A fixed point $x$ is a Lefshetz fixed point of $f$ if $df_x$ has no nonzero fixed point.
		\end{defn}
		So $f$ is a Lefschetz map if and only if all its fixed points are Lefschetz. If $x$ is a Lefschetz fixed point, we denote the orientation number of $(x,x)$ in the intersection $\Delta \cap graph(f)$ by $L_x(f)$, called the local Lefschetz number of $f$ at $x$. Thus for $f$ Lefschetz map,
		\begin{equation}
			L(f) = \sum_{f(x)=x}L_x(f).
		\end{equation}
		$x$ is a Lefschetz fixed point if and only if$df_x - I$ is an isomorphism of $T_x(X)$.
		\begin{prop}
			The local Lefschetz number $L_x(f)$ at a Lefschetz fixed point is $1$ if the isomorphism $df_x-I$ preserves orientation on $T_x(X)$, and $-1$ if the isomorphism reverses orientation. That is the sign of $L_x(f)$ equals the sign of the determinant of $df_x - I$.
		\end{prop}
		We also have this splitting proposition, which enables us to change $f$ locally at an isolated fixed point.
		\begin{prop}(Splitting Proposition)
			Let $U$ be a neighborhood of the fixed point $x$ that contains no other fixed points of $f$. Then there exists a homotopy $f_t$ of $f$ such that $f_t$ has only Lefschetz fixed points in $U$, and each $f_t$ equals $f$ outside some compact subset of $U$.
		\end{prop}
		However, we have a more convenient way to calculate the local Lefschetz number with the following definition.
		\begin{defn}
			Suppose that $x$ is an isolated fixed point of $f$ in $\mathbb{R}^k$. If $B$ is a small closed ball centered at $x$ that contains no other fixed point, then the degree of map
			\begin{equation}
				z \to \frac{f(z)-z}{|f(z)-z|}
			\end{equation}
			is called the local Lefschetz number of $f$ at $x$, denoted $L_x(f)$.
		\end{defn}
		\begin{prop}
			At Lefschetz fixed points, the two definitions of $L_x(f)$ agree.
		\end{prop}
		Again since Lefschetz number is homotopy invariant, we have:
		\begin{prop}
			Suppose that the map $f$ in $\mathbb{R}^k$ has an isolated fixed point at $x$, and let $B$ be a closed ball around $x$ containing no other fixed point of $f$. Choose any map $f_1$ that equals $f$ outside some compact subset of $Int(B)$ but has only Lefschetz fixed points in $B$. Then
			\begin{equation}
				L_x(f) = \sum_{f_1(z)=z}L_z(f_1),
			\end{equation}
			for any $z \in B$.
		\end{prop}
		
		Since any manifold looks like $\mathbb{R}^n$ locally, we can do this inductively and generalize to the case for any compact manifold $X$ and $Y$.						\begin{theorem}(Local Computation of the Lefschetz Number).
			Let $f:X \to Y$ be any smooth map on a compact manifold, with only finitely many fixed points. Then the global Lefschetz number equals the sum of the local Lefschetz numbers:
			\begin{equation}
				L(f) = \sum_{f_x(x)}L_x(f).
			\end{equation}
			
		\end{theorem}
		
	\section{Examples}
		Let's look at some examples.\\
		\indent Since we know the Lefschetz number is just Euler characteristic when the map is homotopic to the identity map, we can compute Euler characteristic by calculating Lefschetz number for some $f$, which is homotopic to identity.
		\begin{exmp}
			The Euler characteristic of $S^2$ is $2$.
		\end{exmp}
		\begin{proof}
			Let $\pi : \mathbb{R}^3 \to S^2$ be the projection $\pi (x) = x/|x|$. Then we define the map $f: S^2 \to S^2$:
			\begin{equation}
				f(x) = \pi (x+(0,0,-1/2)).
			\end{equation}
			This map has a source at $(0,0,1)$, a sink at $(0,0,-1)$ and no fixed point elsewhere, which gives $L(f) = 2$. Also the map $F: S^2 \times I \to S^2$ defined by
			\begin{equation}
				F(x,t) = \pi (x+(0,0,-t/2))
			\end{equation}
			is a homotopy between $f$ and the identity map. By Proposition 1.4, we know that 
			\begin{equation}
				\chi (S^2) = 2.
			\end{equation}
		\end{proof}					
		
		\begin{cor}
			Every map of $S^2$ that is homotopic to the identity must possess a fixed point. In particular, the antipodal map is not homotopic to the identity.
		\end{cor}
		\begin{exmp}
			The surface of genus $k$ admits a Lefschetz map homotopic to the identity, with one source, one sink, and $2k$ saddles. Consequently, its Euler characteristic is $2-2k$.
		\end{exmp}
		
		\begin{exmp}
			The Euler characteristic of a compact connected Lie group is zero.
		\end{exmp}
		\begin{proof}
			Let the compact connected Lie group be $G$ and let $g \in G$ such that $g \neq 1$. Define $f: G \to G$ by $f(x) = g \cdot x$. This smooth map is homotopic to $id_G$ but has no fixed point. Then we know
			\begin{equation}
				\chi (G) = L(f) = 0.
			\end{equation}
		\end{proof}
		
		\section{Homology version}
		There is also a homology version of Lefschetz fixed point theorem.
		\begin{theorem}(The Lefschetz fixed point theorem)
			Let $X$ be a closed smooth manifold and let $f: X \to X$ be a smooth map with all fixed points nondegenerate. Then
			\begin{equation}
				L(f) = \sum_{i} (-1)^i Tr(f_*: H_i(X;\mathbb{Q}) \to H_i(X;\mathbb{Q})).
			\end{equation}
		\end{theorem}
		This theorem is very powerful. For example, we can use this to prove the Brouwer fixed point theorem in 2 lines.
		\begin{exmp}(Brouwer Fixed Point Theorem)
			Every smooth $f: D^n \to D^n$ has a fixed point.
		\end{exmp}
		\begin{proof}
			Since $D^n$ is contractible, we know that the only nontrivial homology group of $D^n$ is $H_0(D^n) = \mathbb{Z}$. Then we know that 
			\begin{equation}
				L(f) = L(f) = \sum_{i} (-1)^i Tr(f_*: H_i(X) \to H_i(X)) = 1.
			\end{equation}
			Then $f$ must have a fixed point.
		\end{proof}
\end{document}